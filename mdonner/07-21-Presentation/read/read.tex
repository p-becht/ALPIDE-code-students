% !TEX root = main.tex
% !TEX program = xelatex

\section*{Slide 1}
\begin{itemize}
    \item One of the first necessary steps in order to operate the chip
	is to determine the threshold of charge, at which the pixels of the
	chip will register an event.
\end{itemize}
\section*{Slide 2}
\begin{itemize}
    \item We're using only a fraction of the chip which serves as a
	representation, since masking the entire chip would take really long.
\end{itemize}
\section*{Slide 3} 
\begin{itemize}
    \item At some charge, the pixel will start firing.
    \item In this region of smearing, assuming the electronic noise is
	gaussian, the hit probablity can be modeled with gaussian
	error function
\end{itemize}
\section*{Slide 4}
\begin{itemize}
    \item After a run has completed, we get the distribution of the thresholds
	of all pixels. We then extract the mean of those thresholds, and 
	repeat the measurement with different settings, to get a reference
	that helps us to determine how the chip has to be configured for 
	our experiment
    \item Just to look at a quick reference,
	In the case of cosmics, we expect an energy deposit of 0.03 MeV in
	the pixel, which is about 7900 e-h-pairs, so in our analysis, what
	we'd \textbf{like} to see, is that the telescope is 100\% efficient.
    \item (Mention VCASN and ITHR)
    \item The error comes from the RMS of the standard deviations
\end{itemize}
\section*{Slide 5} 
\begin{itemize}
    \item Another scan we did is the noiseoccupancy scan, which gives a
	selectable number of random triggers and returns the number of hits.
	If the threshold is too low, some pixels will register a hit, due
	to electronic noise. This is what's called "Fake hit rate"
    \item Now this a measurement taken at no back bias. The sensitivity limit
	is 1 over the amount of times triggered, times the number of pixels.
    \item as you can see the chips fhr deminishes drastically when applying a
	back bias voltage
\end{itemize}
