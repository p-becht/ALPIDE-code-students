% !TEX root = main.tex
% !TEX program = xelatex

\section*{Slide 1}
\begin{itemize}
    \item One of the first necessary steps in order to operate the chip
	is to determine the threshold of charge, at which the pixels of the
	chip will register an event.
\end{itemize}
\section*{Slide 2}
\begin{itemize}
    \item For that we inject a well-defined amount of charge in a number of pixels,
	and see if they fire or not.
    \item We're using only a fraction of the chip which serves as a
	representation, since masking the entire chip would take really long.
    \item These are the parameters we used for the measurement, if you've
	worked with ALPIDE before, you will have seen them somewhere.
    \item If you remember Felicitas' Presentation a few weeks ago, you might
	remember that the chip is seperated into 32 regions, so this value
	means, we are 32 pixels (one per region) \\
	(Since i mention how this works, people who listen or even worked with
	alpide before, will know, and those who dont, wont pay it any mind)
\end{itemize}
\section*{Slide 3} 
\begin{itemize}
    \item For each charge point, we chose to inject 50 times.
    \item If we then increase the Amount, at some point the pixel will start firing.
	This procedure is called an S-Curve scan.
    \item If there was no electronic noise, this would just be a step-function.
	But as you can see, there is a little smearing going on.
	If we now assume that the electronic noise responsible for this smearing is
	gaussian, the hit probablity can be modeled with gaussian error function
\end{itemize}
\section*{Slide 4} 
Just read lol
\section*{Slide 5}
\begin{itemize}
    \item After a run has completed, we get the distribution of the thresholds
	of all pixels. We then extract the mean of those thresholds, and 
	repeat the measurement with different settings, to get a reference
	that helps us to determine how the chip has to be configured for 
	our experiment
    \item Just to look at a quick reference,
	In the case of cosmics, we expect an energy deposit of 29 keV in
	the pixel, which is about 7900 e-h-pairs, so in our analysis, what
	we'd \textbf{like} to see, is that the telescope is 100\% efficient.
    \item The error comes from the RMS of the standard deviations
\end{itemize}
\section*{Slide 6} 
\begin{itemize}
    \item Another scan we did is the noiseoccupancy scan, which gives a
	selectable number of triggers and returns the number of hits.
	If the threshold is too low, some pixels will register a hit, due
	to electronic noise. This is what's called "Fake hit rate"
    \item Now this a measurement taken at no back bias. The sensitivity limit
	is 1 over the amount of times triggered, times the number of pixels.
    \item as you can see the chips fhr deminishes drastically when applying a
	back bias voltage
\end{itemize}
\section*{Slide 7} 
\begin{itemize}
    \item \textbf{Why are there points above the Sensitivity limit?}
	- One reason for that is that our telescope is susceptible for
	cosmics. These measurements can take a while, and it is not
	too unlikely that the sensor is hit by a cosmic muon in the
	meantime. Due to charge-sharing, these muons will be
	registered in multiple pixels at once, hence the 10-ish
	hits.
	- It might look like there are more deviations in this area, but
	actually the density of data points for this measurement taken
	was much higher than the previous one.
\end{itemize}
